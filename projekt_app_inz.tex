% In your .tex file
% !TEX program = pdflatex
\documentclass[11pt,a4paper]{article}
\usepackage{polski}
\usepackage[utf8]{inputenc}
\usepackage{listings}
\usepackage{graphicx}
\usepackage{color}
\usepackage{kpfonts}
\usepackage{graphicx}
\usepackage{lmodern}
\graphicspath{ {C:/Users/Misiek/Desktop/Latex/} }
\usepackage{setspace}
\usepackage[left=2cm,right=2cm,top=2cm,bottom=2cm]{geometry}
\definecolor{grey}{rgb}{0.9,0.9,0.9}
\newcommand{\HRule}[1]{\hfill \rule{0.2\linewidth}{#1}}

\lstset{
	language=[Visual]C++,
	keywordstyle=\bfseries\ttfamily\color[rgb]{0,0,1},
	identifierstyle=\ttfamily,
	commentstyle=\color[rgb]{0.133,0.545,0.133},
	stringstyle=\ttfamily\color[rgb]{0.627,0.126,0.941},
	showstringspaces=false,
	basicstyle=\small,
	numberstyle=\footnotesize,
	numbers=left,
	stepnumber=1,
	numbersep=10pt,
	tabsize=2,
	breaklines=true,
%prebreak = \raisebox{0ex}[0ex][0ex]{\ensuremath{\hookleftarrow}},
	breakatwhitespace=false,
	aboveskip={1.5\baselineskip},
  columns=fixed,
  upquote=true,
  extendedchars=true,
% frame=single,
% backgroundcolor=\color{grey},
}

%\title{Dodatek SW: Aplikacja podająca odchyłki wymiarów wg norm \\PN EN 13920}
%\author{13K1\\Marcin Rędziński\\Michał Zawalski}
\begin{document}
\colorbox{grey}{
	\parbox[t]{1.0\linewidth}{
		\centering \fontsize{30pt}{40pt}\selectfont % The first argument for fontsize is the font size of the text and the second is the line spacing - you may need to play with these for your particular title
		\vspace*{0.7cm} 
		
		\hfill \textbf{Dodatek SW} \\
		\hfill Aplikacja podająca odchyłki \\
		\hfill wymiarów wg norm PN EN 13920 \par
		
		\vspace*{0.7cm} % Space between the end of the title and the bottom of the grey box
	}
}
\vfill
{\centering \large 
\hfill 13K1 \\
\hfill Rędziński Marcin \\
\hfill Zawalski Michał \\
\HRule{1pt}} % Horizontal line, thickness changed here
%\maketitle

\newpage
\tableofcontents
\newpage
\section{Temat oraz wymagania do projektu}

\textbf{Temat: } Dodatek SW: Aplikacja podająca odchyłki wymiarów wg norm PN EN
13920.
\newline
\newline
\textbf{Wymagania do projektu: } Aplikacja ma podawać wartości odchyłek wymiarów
liniowych i kątowych zgodnie z normami dla wskazanego w SW wymiaru i
zadanej klasy dokładności. Dodatkowa funkcja: przypisanie wartości odchyłek
do wskazanego wymiaru.
\newline
\newline
\textbf{Technika wykonania aplikacji: } C\#, SW Add-in

\section{Opis aplikacji}

Aplikacja jest w pełni działającym dodatkiem do programu Solid Works. GUI aplikacji zostało wykonany w Windows Forms co czyni ją dodatkiem hybrydowym w SW. Aplikacja ma za zadanie podawanie wartości odchyłek liniowych i kątowych według norm PN EN 13920. Wartości te można zmieniać według własnych potrzeb a następnie przypisać daną/e odchyłki do zaznaczonego wymiaru modelu w SW. Aplikacja została w pełnym stopniu zabezpieczona przed niewłaściwym działaniem dzięki obsłudze wyjątków instrukcjami (try catch). 

\newpage
\section{Opis elementów aplikacji}
\begin{figure}[h!]
\centering
\includegraphics[scale=1]{form1on.jpg}
\caption{Aktywne okno Form1}
\end{figure}
\textbf{Elementy: }
\newline
\newline
\textit{1} - label1: "Typ wymiaru: "
\newline
\newline
\textit{2} - label2: element pokazujący typ wymiaru ("Liniowy" albo "Kątowy")
\newline
\newline
\textit{3} - label3: "Wartość: "
\newline
\newline
\textit{4} - label4: wartość wskazanego wymiaru (dla Liniowego w milimetrach, dla kątowego w stopniach)
\newline
\newline
\textit{5} - label7: "Klasa dokładności: "
\newline 
\newline
\textit{6} - comboBox1: lista z której możemy wybrać klasę dokładności (A, B, C, D)
\newline
\newline
\textit{7} - label5: "Odchyłka min:
\newline
\newline
\textit{8} - textBox1: proponowana wartość minimalna odchyłki dla danego wymiaru i klasy dokładności wg norm PN EN 13920.
\newline
\newline
\textit{9} - label6: "Odchyłka max" 
\newline
\newline
\textit{10} - textBox2: proponowana wartość maksymalna odchyłki dla danego wymiaru i klasy dokładności wg norm PN EN 13920.
\newline
\newline
\textit{11} - button1: przypisuje ustawione odchyłki do wymiaru w SW
\newline
\newline
\textit{12} - button2: usuwa ustawione wczesniej odchyłki w SW o ile jakieś były
\newline
\newline
\textit{13} - checkBox1: zaznaczony okienko ustawia parametr okienka TopMost na true (domyślnie zaznaczone)
\newline
\newline
\textit{14} - toolStripStatusLabel1: wyświetla informacje co powinien zrobić użytkownik 
\newpage
\begin{figure}[h!]
\centering
\includegraphics[scale=1]{kat.jpg}
\caption{Aktywne okno Form2}
\end{figure}
\textbf{Elementy: }
\newline
\newline
\textit{1} - label1: "Pierwsze ramię kąta "
\newline
\newline
\textit{2} - textBox1: pokazuje długość przypisanego wymiaru liniowego dla pierwszego ramienia kąta (możliwość wprowadzenia wymiaru ręcznie poprzez jego wpisanie)
\newline
\newline
\textit{3} - button1: przypisuje długość zaznaczonego wymiaru do pierwszego ramienia kąta
\newline
\newline
\textit{4} - label2: "Drugie ramię kąta "
\newline
\newline
\textit{5} - textBox2: pokazuje długość przypisanego wymiaru liniowego dla drugiego ramienia kąta (możliwość wprowadzenia wymiaru ręcznie poprzez jego wpisanie)
\newline 
\newline
\textit{6} - button2: przypisuje długość zaznaczonego wymiaru do drugiego ramienia kąta
\newline
\newline
\textit{7} - button3: zamyka okno Form2 przekazując przepisując zatwierdzone długości ramion kąta
\newline
\newline
\textit{8} - button4: zamyka okno Form2
\newline
\newline
\textit{9} - toolStripStatusLabel1: wyświetla informacje co powinien zrobić użytkownik  
\newpage
\section{Fragmenty kodu}
\addcontentsline{toc}{section}{Metoda dodająca przycisk w SW do dodatku w toolboxie}
\section*{Metoda dodająca przycisk w SW do dodatku w toolboxie}
\begin{lstlisting}
cmdIndex0 = cmdGroup.AddCommandItem2("Run Tolerance Tool", -1, "Laduje lub zwalnia dodatek Tolerance Tool", "Tolerance Tool", 0, "RunToleranceTool", "", mainItemID1, menuToolbarOption); 
\end{lstlisting}

Metoda ta zostaje wywołana po dodaniu do naszego dodatku do aktywnych. Dodaje ona przycisk na pasku \textit{C\# Addin} o nazwie \textit{Run Tolerance Tool}. 
\newline
\addcontentsline{toc}{section}{Metoda RunToleranceTool()}
\section*{Metoda RunToleranceTool()}
\begin{lstlisting}
        public void RunToleranceTool()
        {
                if(iSwApp.GetFirstDocument()!=null)
                {
                    form = new Form1(iSwApp);
                    form.Show();
                }
                else
                {
                    System.Windows.Forms.MessageBox.Show("Brak otwartego modelu");
                }
        }
\end{lstlisting}
Metoda wywoływana po naciśnięciu przycisku na pasku zadań w SW. Zawiera ona instrukcje warunkową if else która sprawdza czy jest otwarty dokument w SW. Jeśli tak metoda wywołuje formatkę Form1 z przekazaną zmienną iSwApp typu ISldWorks.
Jeśli nie zostaje pokazany MessageBox z komunikatem "Brak otwartego modelu".
\newline
\addcontentsline{toc}{section}{Konstruktor Form1(ISldWorks iSwApp)}
\section*{Konstruktor Form1(ISldWorks iSwApp)}
\begin{lstlisting}
        public Form1(ISldWorks iSwApp)
        {
            InitializeComponent();
            this.TopMost = true;
            checkBox1.Checked = true;
            Rodzic = iSwApp;

            Thread trd = new Thread(new ThreadStart(this.UpdateUI));
            trd.IsBackground = true;
            trd.Start();
            comboBox1.Text = "A";

        }
\end{lstlisting}
W konstruktorze zostały zdefiniowane początkowe parametry formatki oraz start nowego wątku \emph{trd} który wykonuje metodę w klasie formatki \emph{UpdateUI} by nie wykonywać zawsze wykonywał się tylko jeden wątek co zabezpiecza przez spowolnieniem aplikacji.
\newpage
\addcontentsline{toc}{section}{Klasa ControlExtensions()}
\section*{Klasa ControlExtensions()}

\begin{lstlisting}
    public static class ControlExtensions
    {
        public static void InvokeIfRequired(this Control control, Action action)
        {

            if (control.InvokeRequired)
                control.Invoke(action);
            else
                action();
        }
        public static void InvokeIfRequired<T>(this Control control, Action<T> action, T parameter)
        {
            try
            {
                if (control.InvokeRequired)
                    control.Invoke(action, parameter);
                else
                    action(parameter);
            }
            catch (Exception e)
            {

            }
        }
    }
\end{lstlisting}
Klasa ta posiada dwie metody które służą do przekazywania wartości z jednego wątku do drugiego poprzez zatrzymanie jednego z nich za czas przekazywania danej wartości a następnie wznowienie go.
\newline
\addcontentsline{toc}{section}{Metoda UpdateUI()}
\section*{Metoda UpdateUI()}

W metodzie UpdateUI zostaje obsłużone cykliczne (dzięki zastosowaniu pętli while(true)) przekazywanie informacji o zaznaczonym wymiarze w SW do aplikacji. Żeby tego dokonać potrzebne jest wpierw sprawdzenie instrukcją warunkową if else czy wpierw został wybrany jakiś model:
\newline
\begin{lstlisting}
model = (ModelDoc2)Rodzic.ActiveDoc;
                    if (model != null)
                    {
                        swModelDocExt = model.Extension;
                        selectionMgr = model.SelectionManager;       
\end{lstlisting}
wtedy zostaje przypisany SelectionManager dziedziczący z modelu do obiektu selectionMgr klasy ISelectionMgr.
Następnie sprawdzamy czy wybrany element jest wymiarem:
\begin{lstlisting}
if (selectionMgr.GetSelectedObjectType3(1, 0) == 14)       
\end{lstlisting}
\newpage
a następnie zostają aktywowane elementy okienka:
\begin{lstlisting}
this.InvokeIfRequired((value) => toolStripStatusLabel1.Text = value, "");
this.InvokeIfRequired((value) => comboBox1.Enabled = value, true);
this.InvokeIfRequired((value) => textBox1.Enabled = value, true);                  
this.InvokeIfRequired((value) => textBox2.Enabled = value, true);
this.InvokeIfRequired((value) => button1.Enabled = value, true);
this.InvokeIfRequired((value) => button2.Enabled = value, true);
\end{lstlisting}
dalej tworzymy obiekt typu DisplayDimension i rzutujemy na niego dane zaznaczonego wymiaru aby móc sprawdzić jakiego jest typu
\begin{lstlisting}
sel_obj = selectionMgr.GetSelectedObject6(1, 0);                                   
if (sel_obj.GetType() == (int)swDimensionType_e.swLinearDimension || sel_obj.GetType() == (int)swDimensionType_e.swRadialDimension)
{
      this.InvokeIfRequired((value) => label2.Text = value, "Wymiar liniowy");     
}
\end{lstlisting}
w taki sam sposób robimy dla wymiaru kątowego i dla błędnego zaznaczenia. Następnie pobieramy obiekt zawierający wymiar oraz drugi zawierający tolerancje i rozpoczynamy pracę w głównym wątku w celu dostępu do kontrolek winforms
\newline
\begin{lstlisting}
odl = sel_obj.GetDimension(); 
tolerancja = odl.Tolerance; 
this.Invoke((MethodInvoker)delegate
{
	norma = comboBox1.Text;
	if ((label4.Text != Math.Round(odl.Value, 2) + " ")&&(label4.Text != Math.Round(odl.Value, 2) + " mm"))
	{
			if (sel_obj.GetType() == (int)swDimensionType_e.swAngularDimension)
		{
			uruchom_form2 = 1;
			ramie = 0;
			textBox1.Text = textBox2.Text = "";
		}
		else
			uruchom_form2 = 0;
	}
	else
		uruchom_form2 = 0;
	if (label2.Text == "Wymiar liniowy")
		label4.Text = Math.Round(odl.Value, 2) + " mm";
	if (label2.Text == "Wymiar katowy")
		label4.Text = Math.Round(odl.Value, 2) + " ";});
\end{lstlisting}
\newpage
w wątku powyżej w przypadku wymiaru kątowego ustawiamy 
flage uruchom\_form2 na wartość 1 w celu późniejszego
 pobrania brakujących wymiarów boków przystających a 
 następnie wyświetlamy w elemencie Form1 label4 
 zaokrąglony wybrany wymiar do 2 miejsc po przecinku.
 Poniżej zostanie pokazany kod obsługujący zlecenie pobrania wymiaru z drugiej formatki:
\begin{lstlisting}
if (uruchom_form2 == 1)
{
	sel_obj_b = sel_obj;
	form2 = new Form2(Rodzic);
	form2.ShowDialog();
	if (form2.DialogResult == DialogResult.Cancel)
	{
		ramie = 0;
	}
	else
	{
		ramie = form2.ramie;
		form2.Dispose();
		this.Invoke((MethodInvoker)delegate
		{
			Update_Tolerance();
		});
	}
	model.ClearSelection();
	s_data = selectionMgr.CreateSelectData();
	selectionMgr.AddSelectionListObject(sel_obj_b, s_data);
}
\end{lstlisting}
W przypadku gdy nie jest zaznaczony żaden prawidłowy wymiar główna formatka zostaje "zablokowana" i czeka na wskazanie poprawnego wymiaru.
\begin{lstlisting}
	else
	{
		this.InvokeIfRequired((value) => toolStripStatusLabel1.Text = value, "Zaznacz prawidlowy wymiar");
		this.InvokeIfRequired((value) => label2.Text = value, "Brak zaznaczenia");
		this.InvokeIfRequired((value) => label4.Text = value, "0");
		this.InvokeIfRequired((value) => comboBox1.Enabled = value, false);
		this.InvokeIfRequired((value) => textBox1.Enabled = value, false);
		this.InvokeIfRequired((value) => textBox2.Enabled = value, false);
		this.InvokeIfRequired((value) => textBox1.Text = value, "");
		this.InvokeIfRequired((value) => textBox2.Text = value, "");
		this.InvokeIfRequired((value) => button1.Enabled = value, false);
		this.InvokeIfRequired((value) => button2.Enabled = value, false);
	}
\end{lstlisting}
Główny wątek zostaje uśpiony na 100ms gdyż częstsze wykonywanie nie jest konieczne a oszczędzamy dzięki temu procesor.
\begin{lstlisting}
Thread.Sleep(100);
\end{lstlisting}
\newpage
\addcontentsline{toc}{section}{Metoda Update\_Tolerance()}
\section*{Metoda Update\_Tolerance()}
\begin{lstlisting}
void Update_Tolerance()
        {
            double wymiar;
            textBox1.Text = textBox2.Text = "0";
            if (label2.Text == "Wymiar liniowy")
            {
                wymiar = Math.Round(odl.Value, 2);
                switch (norma)
                {
                    case "A":
                        if (wymiar < 2.0)
                        {
                            textBox1.Text = textBox2.Text = "0";
                        }
                        else if (wymiar >= 2.0 && wymiar <= 400)
                        {
                            textBox1.Text = textBox2.Text = "1";
                        }
\end{lstlisting}
Funkcja to proponuje odchyłki wymiarów wg norm PN EN 13920 w zależności od długości zaznaczonego wymiaru:
\newpage
\section{Podręcznik użytkowania}
Po dodaniu naszego dodatku do przestrzeni roboczej SolidWorksa nie musimy go dodawać do aktywnych ponieważ własne dodatki są automatycznie dodawane do paska. Wybieramy go z zakładki \textbf{C\# Addin}.
\begin{figure}[h!]
\centering
\includegraphics[scale=0.7]{3.jpg}
\end{figure}
\newline
Nasz dodatek jest włączony i gotowy do działania.
\newline
W celu dodania odchyłki do wymiaru liniowego wpierw musimy zaznaczyć interesujący nas wymiar lub wybrać klasę dokładności a następnie zaznaczyć wymiar.
\begin{figure}[h!]
\centering
\includegraphics[scale=0.7]{5.jpg}
\end{figure}
\newline
Aplikacja zaproponowała nam górną jak i dolną odchyłkę wg norm PN EN 13920 i wybranej klasy dokładności (domyślnie A). Po wybraniu innej klasy z listy wartości odchyłek zostaną zmienione na nową wartość zgodną z normą. Po naciśnięciu przycisku Przypisz zdefiniowane przez nas wartości odchyłek zostaną przypisane do wymiaru.
\begin{figure}[h!]
\centering
\includegraphics[scale=0.7]{9.jpg}
\end{figure}
\newpage
Procedura do nadania odchyłek dla kąta jest podobna lecz w tym celu dodatkowo musimy zaznaczyć i przypisać dany wymiar lub wprowadzić go ręcznie do 2 ramion kąta. W tym celu po zaznaczeniu wymiaru kątowego wyświetla nam się druga formatka
\begin{figure}[h!]
\centering
\includegraphics[scale=0.7]{6.jpg}
\end{figure}
\newline
Wymiary te można wpisać lub wprowadzić je przez naciśnięcie przycisku Przypisz obok odpowiedniego boku po przy zaznaczonym wymiarze w SW.
\begin{figure}[h!]
\centering
\includegraphics[scale=0.7]{7.jpg}
\end{figure}
\newline
Po zatwierdzeniu wrócimy do głównego okienka aplikacji z zaproponowanymi odchyłkami które możemy zmienić tak samo jak klasę dokładności. Gdy odchyłki zostaną przez Nas dopasowane możemy tak samo jak w przypadku wymiaru liniowego przypisać je do wymiaru.
\begin{figure}[h!]
\centering
\includegraphics[scale=0.7]{8.jpg}
\end{figure}
\newline
\end{document}
